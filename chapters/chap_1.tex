\chapter{Introduction} \label{chap:intro}
Always start a chapter with a short but informative text about the following sections Point out the relevance of the sections and create interconnections between them Never ever just write a single sentence here Furthermore,\ you are strongly advised to respect the hints given in this template

If necessary,\ \emph{emphasize} some words in your text, for example technical terms like \emph{Cross-Site Scripting}.
It can help readers to highlight essentials parts for a better understanding.
But take care to not highlight too much.

\section{Motivation}
What is the motivation to deal with this subject?
Why is this topic interesting and relevant from research perspective?
Which interesting problems do you expect?

Do not abbreviate \enquote{e.\,g.\/} within a sentence,\ always write \enquote{for example} However,\ within in parentheses you are allowed to abbreviate and use,\ e.\,g.,\ and, i.\,e.,\ as shown here: with a comma right before and after it In addition to that,\ ensure correct spacing by using \texttt{\textbackslash,} in between

\section{Related Work}

In the introductory section, provide a concise yet informative summary of the most significant previous works related to this thesis. Mention two to four research papers that bear close resemblance in terms of objectives or methodological approach. These references will serve as valuable context for the reader and help establish the relevance of your study within its respective field.

\section{Contribution}
\label{sec:Contribution}

When writing the Introduction section of your thesis, it is important to include a contributions subsection where you can directly state the key results and contributions of your work. Unlike a novel or thriller, your thesis is not meant to keep the reader in suspense. It is important to declare your results and key contributions directly in the introduction.

One common mistake is to hide the interesting parts of your work in the introduction, hoping to build suspense and keep the reader engaged. This is not necessary in academic writing. Instead, you should clearly state your key results and contributions so that readers can understand the significance of your work right from the start.

The contributions subsection should provide a clear and concise summary of what you have achieved in your research. You should highlight the new knowledge or insights that your work has contributed to the field, as well as any practical applications or implications. This is an opportunity to convince readers that your work is important and worth their time.

Remember that the contributions subsection should not be overly technical or detailed. Its purpose is to provide a high-level overview of your work and its significance, not to get into the nitty-gritty of your research methods or results. Keep your writing clear and concise, focusing on the key contributions that your work has made to the field.

\section{Organization of this Thesis}
Please give a general overview on how your thesis is divided into sections and chapters~\dots

We recommend the following guidelines for the thesis.

\begin{itemize}
    \item Length of the thesis:
    \begin{description}
        \item[Bachelor's theses] should have a length of 30-50 content pages, with a tendency towards 50 pages.
        \item[Master's theses] should have a length of 50-70 content pages, with a tendency towards 50 pages.
    \end{description}
    \item Your thesis must use hyperlinks on references~\cite{AviramSSHDSVAHD16}, \href{https://example.com}{links}, etc.This functionality is activated by default in this template with the package \code{hyperref}, so that you can click on them in the PDF.
    \item Your thesis must include bookmarks in the PDF application. This functionality is activated by default in this template with the package \code{hyperref}, so that you can open the bookmarks panel in you PDF application.
\end{itemize}

If you think that your work should deviate from these guidelines, please contact your supervisor.